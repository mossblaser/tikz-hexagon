\documentclass{article}

\usepackage{fullpage}
\usepackage{ifthen}

\usepackage{tikz}
\usetikzlibrary{hexagon,positioning}

\title{Using the \texttt{hexagon} TikZ Library}
\author{Jonathan Heathcote}
\date{}

\newcommand{\example}[2]{%
		\begin{tikzpicture}[thick]
			
			\node (example) [ fill=yellow!30!white
			                ]
			{
				#1
			};
			
			\node (source) [ text width=\textwidth
			               , anchor=top left
			               , right=1ex of example.north east
			               ]
			{
					#2
			};
			
		\end{tikzpicture}%
}


\begin{document}
	
	\maketitle
	
	\section{Introduction}
		
		Drawing diagrams of hexagonal structures for presentations and papers is
		pretty dull, laborious work. Having recently started playing with TikZ it
		occurred to me that it would be nice to have some aids for working with
		hexagons in the library. The easiest way to install this library is to copy
		it into the working directory of your document\footnote{Though it should
		really go in
		\texttt{\$HOME/texmf/tex/generic/pgf/frontendlayer/tikz/libraries/tikzlibraryhexagon.code.tex}}\ldots{}
		Before you get started you'll need to enable the library:
		
		\begin{verbatim}
			\usetikzlibrary{hexagon}
		\end{verbatim}
	
	\section{Drawing Hexagons}
		
		In order to draw a hexagon, simply add the \texttt{hexagon} attribute to a
		node:
		\vspace{1em}
		
		\begin{tikzpicture}[thick]
			\node [draw, hexagon] {};
		\end{tikzpicture}
		\begin{verbatim}
			\begin{tikzpicture}[thick]
			  \node [draw, hexagon] {};
			\end{tikzpicture}
		\end{verbatim}
		
		The size of the hexagon can be specified using the \texttt{minimum width}
		attribute:
		\vspace{1em}
		
		\begin{tikzpicture}[thick]
			\node [draw, hexagon, minimum width=2cm] {};
			
			\draw [<->,red]
			      (-1cm,0) -- node [above] {2cm} ++(2cm,0);
		\end{tikzpicture}
		\begin{verbatim}
			\begin{tikzpicture}[thick]
			  \node [draw, hexagon, minimum width=2cm] {};
			  
			  \draw [<->,red]
			        (-1cm,0) -- node [above] {2cm} ++(2cm,0);
			\end{tikzpicture}
		\end{verbatim}
		
		Anchors have been added to each of the edges:
		
		\begin{tikzpicture}[hexagonXYZ,thick]
			\node (n) [draw,hexagon,minimum size=1cm] {};
			
			\draw (n.side north)      -- ++(0,0.1,0)  -- ++(-1,0) node [left]
			  {\texttt{side north}};
			\draw (n.side west)                       -- ++(-1,0) node [left]
			  {\texttt{side west}};
			\draw (n.side south west) -- ++(0,0,0.1)  -- ++(-1,0) node [left]
			  {\texttt{side south west}};
			\draw (n.side north east) -- ++(0,0,-0.1) -- ++( 1,0) node [right]
			  {\texttt{side north east}};
			\draw (n.side east)                       -- ++( 1,0) node [right]
			  {\texttt{side east}};
			\draw (n.side south)      -- ++(0,-0.1,0) -- ++( 1,0) node [right]
			  {\texttt{side south}};
		\end{tikzpicture}
	
	
	\section{Positioning Hexagons}
		
		To allow easy positioning of hexagons, an attribute \texttt{hexagonXYZ} is
		provided which can be added to an environment to map all ($x$,$y$) and
		($x$,$y$,$z$) coordinates into the hexagonal space.
		
		\vspace{1em}
		
		\begin{tikzpicture}[hexagonXYZ,thick]
		  \coordinate (axis) at (-3, 1.5, 0);
		  \draw [->] (axis) -- ++(1,0,0) node [right] (x) {$x$};
		  \draw [->] (axis) -- ++(0,1,0) node [left]  (y) {$y$};
		  \draw [->] (axis) -- ++(0,0,1) node [left]  (z) {$z$};
			
			\foreach \x in {0,...,3}{
				\foreach \y in {0,...,3}{
					\node [draw,hexagon,minimum width=1cm,font=\scriptsize,inner sep=0]
						at (\x,\y)
						{(\x,\y)};
				}
			}
		\end{tikzpicture}
		\begin{verbatim}
		\begin{tikzpicture}[hexagonXYZ,thick]
		  
		  \coordinate (axis) at (-3, 1.5, 0);
		  \draw [->] (axis) -- ++(1,0,0) node [right] (x) {$x$};
		  \draw [->] (axis) -- ++(0,1,0) node [left]  (y) {$y$};
		  \draw [->] (axis) -- ++(0,0,1) node [left]  (z) {$z$};
		  
		  \foreach \x in {0,...,3}{
		    \foreach \y in {0,...,3}{
		      \node [draw,hexagon,minimum width=1cm,font=\scriptsize,inner sep=0]
		        at (\x,\y)
		        {(\x,\y)};
		    }
		  }
		  
		\end{tikzpicture}
		\end{verbatim}
	
	\section{Alternative Rotation}
		
		By default hexagons are oriented point-up (as used for the layout of chips
		in SpiNNaker). An alternative flat-edge-up form, \texttt{hexagonBoard}, is
		provided along with the associated coordinate system,
		\texttt{hexagonBoardXYZ}:
		
		\vspace{1em}
		
		\begin{tikzpicture}[hexagonBoardXYZ,thick]
		  \coordinate (axis) at (-3, 1.5, 0);
		  \draw [->] (axis) -- ++(1,0,0) node [right] (x) {$x$};
		  \draw [->] (axis) -- ++(0,1,0) node [left]  (y) {$y$};
		  \draw [->] (axis) -- ++(0,0,1) node [left]  (z) {$z$};
			
			\foreach \x in {0,...,3}{
				\foreach \y in {0,...,3}{
					\node [draw,hexagonBoard,minimum width=1cm,font=\scriptsize,inner sep=0]
						at (\x,\y)
						{(\x,\y)};
				}
			}
		\end{tikzpicture}
		\begin{verbatim}
		\begin{tikzpicture}[hexagonBoardXYZ,thick]
		  \coordinate (axis) at (-3, 1.5, 0);
		  \draw [->] (axis) -- ++(1,0,0) node [right] (x) {$x$};
		  \draw [->] (axis) -- ++(0,1,0) node [left]  (y) {$y$};
		  \draw [->] (axis) -- ++(0,0,1) node [left]  (z) {$z$};
		  
		  \foreach \x in {0,...,3}{
		    \foreach \y in {0,...,3}{
		      \node [draw,hexagonBoard,minimum width=1cm,font=\scriptsize,inner sep=0]
		        at (\x,\y)
		        {(\x,\y)};
		    }
		  }
		\end{tikzpicture}
		\end{verbatim}
	
	\break
	\section{Drawing SpiNNaker Boards}
		
		For the moment I can't figure out how to put the calculation of where the
		nodes go into TikZ but here's something temporary I generated with Python...
		
		% from topology import hexagon
		% 
		% hex_points = dict(((x,y), num) for num, (x,y) in enumerate(hexagon()))
		% 
		% xs = set(x for x,y in hex_points.keys())
		% ys = set(y for x,y in hex_points.keys())
		% 
		% for y in range(min(ys), max(ys)+1)[::-1]:
		%   for x in range(min(xs), max(xs)+1):
		%     if (x,y) in hex_points:
		%       print ("%s/%s/%s,"%(x,y,hex_points[(x,y)] + 1)).ljust(9),
		%     else:
		%       print " "*9,
		%   
		%   print "%"
		
		\vspace{1em}
		
		\begin{tikzpicture}[hexagonXYZ,thick]
			
		  \foreach \x/\y/\num in {%
		                                            0/4/45,   1/4/46,   2/4/47,   3/4/48, %
		                                  -1/3/44,  0/3/25,   1/3/26,   2/3/27,   3/3/28, %
		                        -2/2/43,  -1/2/24,  0/2/11,   1/2/12,   2/2/13,   3/2/29, %
		              -3/1/42,  -2/1/23,  -1/1/10,  0/1/3,    1/1/4,    2/1/14,   3/1/30, %
		    -4/0/41,  -3/0/22,  -2/0/9,   -1/0/2,   0/0/1,    1/0/5,    2/0/15,   3/0/31, %
		    -4/-1/40, -3/-1/21, -2/-1/8,  -1/-1/7,  0/-1/6,   1/-1/16,  2/-1/32,          %
		    -4/-2/39, -3/-2/20, -2/-2/19, -1/-2/18, 0/-2/17,  1/-2/33,                    %
		    -4/-3/38, -3/-3/37, -2/-3/36, -1/-3/35, 0/-3/34,                              %
		  }{
		    \node [draw,hexagon,minimum size=1cm,text=gray!50!white]
		          at (\x,\y) (spinnaker chip \num){\num};
		  }
		  
		  % Label two chips (covering the number)
		  \node at (spinnaker chip 1)  [red,fill=white,circle] (A) {A};
		  \node at (spinnaker chip 43) [red,fill=white,circle] (B) {B};
		  
		  % Draw the shortest path between them
		  \draw [->, red, very thick]
		        (A) -- (spinnaker chip 9.center) -- (B);
			
		\end{tikzpicture}
		
		\begin{verbatim}
		\begin{tikzpicture}[hexagonXYZ,thick]
		  \foreach \x/\y/\num in {%
		                                            0/4/45,   1/4/46,   2/4/47,   3/4/48, %
		                                  -1/3/44,  0/3/25,   1/3/26,   2/3/27,   3/3/28, %
		                        -2/2/43,  -1/2/24,  0/2/11,   1/2/12,   2/2/13,   3/2/29, %
		              -3/1/42,  -2/1/23,  -1/1/10,  0/1/3,    1/1/4,    2/1/14,   3/1/30, %
		    -4/0/41,  -3/0/22,  -2/0/9,   -1/0/2,   0/0/1,    1/0/5,    2/0/15,   3/0/31, %
		    -4/-1/40, -3/-1/21, -2/-1/8,  -1/-1/7,  0/-1/6,   1/-1/16,  2/-1/32,          %
		    -4/-2/39, -3/-2/20, -2/-2/19, -1/-2/18, 0/-2/17,  1/-2/33,                    %
		    -4/-3/38, -3/-3/37, -2/-3/36, -1/-3/35, 0/-3/34,                              %
		  }{
		    \node [draw,hexagon,minimum size=1cm,text=gray!50!white]
		          at (\x,\y) (spinnaker chip \num){\num};
		  }
		  
		  % Label two chips (covering the number)
		  \node at (spinnaker chip 1)  [red,fill=white,circle] (A) {A};
		  \node at (spinnaker chip 43) [red,fill=white,circle] (B) {B};
		  
		  % Draw the shortest path between them
		  \draw [->, red, very thick]
		        (A) -- (spinnaker chip 9.center) -- (B);
		  
		\end{tikzpicture}
		\end{verbatim}
	
\end{document}
